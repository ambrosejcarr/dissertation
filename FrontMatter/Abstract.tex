% This is the abstract of my dissertation.

\pagestyle{empty} % No page number in entire abstract
\begin{center}
  ABSTRACT

  A Single-Cell Immune Map of Normal and Cancerous Breast Reveals an Expansion of Phenotypic States Driven by the Tumor Microenvironment

  Ambrose J. Carr
\end{center}

Knowledge of the phenotypic states of immune cells in the tumor microenvironment is essential to understand of immunological mechanisms of cancer progression, responses to cancer immunotherapy, and the development of novel rational treatments.
Yet, this knowledge is opaque to traditional bulk sequencing methods, and novel single-cell RNA sequencing (scRNA-seq) methods which could address these questions introduce complex patterns of error into data that are poorly characterized. 
This dissertation describes a computational framework, SEQC, built to facilitate rapid and agile analysis of scRNA-seq approaches that utilize molecular barcodes.
It combines SEQC with a clustering and normalization method, BISCUIT, and approaches to examine phenotypic diversity and gene variation. 
These methods are applied to address the unique computational challenges inherent to analysis of single-cell RNA-seq data derived from multiple patients with diverse phenotypes. 
Using these computational approaches, it executes an experiment comprising scRNA-seq of over 47,000 immune cells collected from primary breast carcinomas, matched normal breast tissue, peripheral blood, and lymph node, from eight patients.  
This atlas revealed significant similarity between normal and tumor tissue resident immune cells.
However, it also describes continuous tumor-specific phenotypic expansions driven by distinct environmental cues.
These results argue against discrete activation states in T cells and the polarization model of macrophage activation in cancer and have important implications for characterizing tumor-infiltrating immune cells.
