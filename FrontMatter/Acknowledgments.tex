% This is the acknowledgments page of my

\cleartorecto % A memoir-class command for moving the acknowledgments to a recto page, not verso.
\chapter{Acknowledgements} % For the heading on the page, also registers in the table of contents
\thispagestyle{plain} %This page should have numbers.

Before beginning my PhD, I had never written a single line of code. 
My studies had woven through Neuroscience and Molecular Biology, but my contributions were exclusively made at the lab bench.  
To this day I have no idea what Dana saw in me that suggested that I would be successful in a computational PhD, but I am indebted to her for the patience she has shown and mentoring she has provided. 
Always available to provide advice and right my course when I digressed, these projects would not have been possible without Dana's support, insight, and connections.

I also want to thank Prof. Aviv Regev and now-professors Dr. Rahul Satija and Dr. Alex Shalek, who, through generosity of time and expertise, both helped shape my impressions of nascent single-cell sequencing, and also opened my eyes to the open questions, motivating much of the work contained in this dissertation. 

Like many dissertation, this one summarizes work that contains contributions from many researchers. 
None of these experiments would be possible if not for the hard work of my tireless wet-lab colleagues: Dr. Linas Mazutis, Juozas Nainys, and Vaidotas Kiseliovas, who carried out all of the sequencing experiments described, were instrumental in providing feedback on the computational approaches I developed, and implemented the modifications to the barcodes and InDrop procedures discussed in Chapter 2. 
Similarly, Thank you to Dr. Manu Setty, Dr. The Phuong Dao, and Kristy Choi for their programming contributions to SEQC, and to George Plitas, the surgeon who obtained all of the samples for these experiments, and who provided an abundance of biological insight. 
Thanks to Sandhya Prabhakaran and Elham Azizi, who transformed my simple ideas about how to combine data from disparate scRNAseq patient samples into a rich Bayesian hierarchical model that I would never have been able to construct myself. Their contributions made much of the work discussed in Chapter 3 possible.  
Finally, thank you to Elham Azizi and Andrew Cornish for their help with the analysis of the processed Breast Cancer Immune Atlas, described in Chapter 4.  

As no experiments can proceed in a vacuum, I am grateful for the insight and criticism provided by my committee members Prof. Harmen Bussemaker and Prof. Peter Sims, and for Prof. Rahul Satija and Prof. Alexander Rudensky, for serving on my defense committee.    
I am also grateful to the Howard Hughes Medical Institute, who generously funded me during my PhD at Columbia University, and to all of my collaborators: Prof. Peter Sims, Prof. Alexander Rudensky, Prof. Frederic Geissmann, Dr. George Plitas, and Kasia Konnopacki for pushing my research in directions I would not have explored on my own.  
Similarly, I am grateful to Dr. Elham Azizi, Dr El-ad David Amir, Dr. Jacob Levine, Dr. Sandhya Prabhakaran, Andrew Cornish, Dr. Manu Setty, Cassandra Burdziak, Roshan Sharma, and Kristy Choi for their help editing this dissertation and for their insight and friendship, which made coming to work easy.  

Finally, thank you to my family, and my wife, Sharonmoyee Goswami, for their support throughout my PhD. 

