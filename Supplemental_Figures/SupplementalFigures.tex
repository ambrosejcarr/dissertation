
\chapter[Supplementary Figures][Supplementary Figures]{Supplementary Figures}

\begin{figure}
\centering
\includegraphics[width=\textwidth]{FigureS1-B.png}
\caption{t-SNE projection of complete immune systems from six breast cancer tumors.\ scRNA-seq data for each tumor is processed with pipeline in Figure S1B and library size-normalized; each dot represents a single- cell colored by PhenoGraph clustering, and clusters are labeled by inferred cell types. Two additional tumors are presented in~\ref{fig:1c}.
}
\label{fig:s1b}
\end{figure}

\begin{figure}
\centering
\includegraphics[width=0.75\textwidth]{FigureS1-D.png}
\caption{Expression of metabolic signatures: fatty acid metabolism (E), glycolysis (F), and phosphorylation (G), summarized as boxplots (left) showing expression of each respective signature (defined as the mean normalized expression of genes in each signature listed in Table S4) across immune cells from each patient; and heatmap (right) displaying z-scored mean expression of genes in each signature; (top) barplot showing total expression of each gene indicated in the heatmap across all patients. See Figure~\ref{fig:1e} for one additional signature
}
\label{fig:s1d}
\end{figure}

\begin{figure}
\centering
\includegraphics[width=\textwidth]{FigureS2-A.png}
\caption{(A) Posterior probability of assignment of cells to clusters in the Biscuit model in the full immune cell atlas of combined tissues and patients presented in Figure 2E; note broad distributions in assignment of naive T cells (bottom) as compared to other cell types.
}
\label{fig:s2a}
\end{figure}

\begin{figure}
\centering
\includegraphics[width=\textwidth]{FigureS2-F.png}
\caption{Distribution of Biscuit alpha parameters per cell vs log of library size, with cells colored by clusters; Biscuit alpha parameters correct for differences in library size across and within clusters.
}
\label{fig:s2f}
\end{figure}

\begin{figure}
\centering
\includegraphics[width=\textwidth]{FigureS2-G.png}
\caption{Distribution of inferred cell-specific parameters alpha and beta in Biscuit across cells from each patient. These differences were corrected in normalizing with alpha and beta parameters.
}
\label{fig:s2g}
\end{figure}

\begin{figure}
\centering
\includegraphics[width=\textwidth]{FigureS2-C.png}
\caption{Robustness analysis of clusters performed with 10-fold cross-validation; boxplots summarize the probability of a pair of cells being assigned to the same final cluster across all 10 subsets.
}
\label{fig:s2c}
\end{figure}

\begin{figure}
\centering
\includegraphics[width=\textwidth]{Figure2-G.png}
\caption{Histogram of frequency of patients contributing to each cluster showing that 19 clusters (out of 95) are present in all 8 patients and 10 clusters are patient-specific.
}
\label{fig:2g}
\end{figure}

\begin{figure}
\centering
\includegraphics[width=\textwidth]{FigureS2-D.png}
\caption{Histogram of frequency of patients contributing to each cluster showing that 19 clusters (out of 95) are present in all 8 patients and 10 clusters are patient-specific.
}
\label{fig:s2d}
\end{figure}

\begin{figure}
\centering
\includegraphics[width=\textwidth]{FigureS2-B.png}
\caption{Bhattacharyya pairwise distances between clusters of Figure~\ref{fig:2h} (blue: small distance to yellow: large distance).
}
\label{fig:s2b}
\end{figure}


\begin{figure}
\centering
\includegraphics[width=\textwidth]{FigureS2-E.png}
\caption{Left: Violin plot of pairwise Bhattacharyya distances between distribution of expression of each gene between all pairs of clusters in the same or different cell types considering mean and covariance of expression, averaged across all genes.
Right: same as left, but after removing the effect of cluster mean in computing similarity, thus considering only covariance.
}
\label{fig:s2e}
\end{figure}

\begin{figure}
\centering
\includegraphics[width=\textwidth]{FigureS3.png}
\caption{Hallmark GSEA enrichment results on genes with highest difference in variance in tumor vs normal tissue in (A) NK and (B) monocytic cells. See Figure 3E for enrichment in T cells; complete lists of enrichments are presented in Table S5.
}
\label{fig:s3}
\end{figure}


\begin{figure}
\centering
\includegraphics[width=\textwidth]{FigureS4-A.png}
\caption{Hartigan’s dip test on density of cells projected on diffusion components, showing statistically significant continuity (lack of “dips”) in cells along T cell activation component (component 3, third panel from left), whereas other components exhibit more defined states (multimodality).
}
\label{fig:s4a}
\end{figure}

\begin{figure}
\centering
\includegraphics[width=\textwidth]{FigureS4-B.png}
\caption{Violin plot of cells projected on terminal differentiation diffusion component: terminal differentiation components organized by tissue type (left panels) and cluster (center panel). Also, heatmap showing expression of immune-related genes with the largest positive correlations with component, averaged per cluster and z-score standardized across clusters; columns (clusters) are ordered by mean projection along the component.
}
\label{fig:s4b}
\end{figure}

\begin{figure}
\centering
\includegraphics[width=\textwidth]{FigureS4-D.png}
\caption{Visualization of all T Cell clusters using activation (component 3), terminal differentiation (component 4), and tissue specificity (component 6) diffusion components. Cells are colored by clusters, and by tissue type in insert. The main trajectories are indicated with arrows and annotated with labels.
}
\label{fig:s4d}
\end{figure}


\begin{figure}
\centering
\includegraphics[width=\textwidth]{FigureS5-A.png}
\caption{Hartigan’s dip test on density of cells projected on diffusion components indicating no diffusion components across myeloid cells show statistically significant continuity, implying myeloid cells reside in defined (multimodal) states along major components explaining variation.
}
\label{fig:s5a}
\end{figure}

\begin{figure}
\centering
\includegraphics[width=\textwidth]{FigureS5-B.png}
\caption{Heatmap showing expression of immune-related markers with the largest positive correlation with TAM activation, pDCs, and monocyte activation components.
}
\label{fig:s5b}
\end{figure}

\begin{figure}
\centering
\includegraphics[width=\textwidth]{FigureS6-A.png}
\caption{Hartigan’s dip test on density of cells projected on diffusion components indicating no diffusion components across myeloid cells show statistically significant continuity, implying myeloid cells reside in defined (multimodal) states along major components explaining variation.}
\label{fig:s6a}
\end{figure}

\begin{figure}
\centering
\includegraphics[width=\textwidth]{FigureS6-B.png}
\caption{Heatmap showing expression of immune-related markers with the largest positive correlation with TAM activation, pDCs, and monocyte activation components.
}
\label{fig:s6b}
\end{figure}


\begin{figure}
\centering
\includegraphics[width=\textwidth]{FigureS6-C.png}
\caption{Violin plot showing the density of cells projected along pDC component and organized by tissue type and cluster.
}
\label{fig:s6c}
\end{figure}


\begin{figure}
\centering
\includegraphics[width=\textwidth]{FigureS7-A.png}
\caption{Displaying null distributions and observed covariances between MACRO and CD276 in raw, unnormalized data using hypothesis testing, subsampling, and permutation (see STAR methods), showing that the differences in covariance in normalized data as shown in  Figure 7B are also present in un-normalized and un-imputed data, and hence is not an artifact of computation.
Bivariate plots of expression levels of MARCO and CD276 in Treg clusters based on inferred mean and covariance parameters from Biscuit. Dark blue color indicates the highest density of cells and light yellow the lowest density of cells.
}
\label{fig:s7a}
\end{figure}

